\section{Details of Numerical Experiments}\label{sec:app-numerical-details}
\paragraph{Details for Example~\ref{ex:firm-benefit-hurt} and Figure~\ref{fig:firm-benefit-hurt-dist}}For the scenario where the firm is negatively affected by the biased response is Example~\ref{ex:firm-benefit-hurt} we used $\vmu_1^T=(2, 4)$ and $\vmu_0^T=(2, 3)$ with $\Sigma_1=\begin{psmallmatrix}0.5 & 0 \\ 0 & 0.5 \end{psmallmatrix}$ and $\Sigma_0=\begin{psmallmatrix}1 & 0.5 \\ 0.5 & 1 \end{psmallmatrix}$, and we multiplied the generated data by 10. For the scenario where the firm benefits from agents' biased response we let $\vmu_1^T=(3, 5)$ and let the rest of the parameters be the same as the first scenario, i.e., $\vmu_0^T=(2, 3)$ with $\Sigma_1=\begin{psmallmatrix}0.5 & 0 \\ 0 & 0.5 \end{psmallmatrix}$ and $\Sigma_0=\begin{psmallmatrix}1 & 0.5 \\ 0.5 & 1 \end{psmallmatrix}$, and we multiplied the generated data by 10. In both scenarios, we let $B=5$. 

We used the Prelec function described in Section~\ref{sec:model} for the behavioral response. Solving the optimization problem takes a considerable amount of time for a large number of data points, here $20,000$, so we used the equivalent of the optimization problem for agents' movement and dictated the movement straight to each data point instead of solving the optimization.

To model agents' behavioral responses, we first identified the agents that would attempt to manipulate their features. Then, we used the movement function with the specified mode, either ``B'' or ``NB'', to move the data points and create a new dataset for post-response. 

For the last row of Figure~\ref{fig:firm-benefit-hurt-dist} we used $\vmu_1^T=(4, 4)$ and $\vmu_0^T=(2, 3)$ with $\Sigma_1=\begin{psmallmatrix}1 & 0 \\ 0 & 1 \end{psmallmatrix}$ and $\Sigma_0=\begin{psmallmatrix}3 & 0 \\ 0 & 1 \end{psmallmatrix}$, and we multiplied the generated data by 10. We used $B=10$.

\paragraph{Details for Figure~\ref{fig:BR-illustration}, Figure~\ref{fig:BR-illustration-quad-cost}, and Figure~\ref{fig:BR-illustration-lin-cost}} We generated 150 data points using different distributions for each feature. Feature 1 was sampled from $\mathcal{N}(700, 200)-\mathcal{D}((0, 20, 50, 100),(0.6, 0.2, 0.1, 0.1))$ where the second term is a discrete distribution selecting 0 with $p=0.6$, 20 with $p=0.2$, 50 with $0.1$, and 100 with $p=0.1$. Feature 2 was sampled from $1500-\Gamma(4, 100)$. We used a $Score$ column to label each individual for later. The score was calculated from the feature weights $(0.65, 0.35)$. We then used a sigmoid function to assign approval probability and label the sampled data points: $\frac{1}{1+\exp(-0.8\times (\frac{x}{10}-80))}$. We assigned the labels using the calculated approval probability and a random number generator. After generating the dataset, we used two copies, one for behavioral response and one for non-behavioral response. 

In Figure~\ref{fig:BR-illustration}, for agents' response to the algorithm, we calculated the agents that can afford the response with a budget of $B=100$ and performed an optimization problem on only those agents. We solved a cost minimization problem for each agent in the band specified by Lemma~\ref{lemma:band-optimization}: $\argmin_\vx cost=\norm{\vx-\vx_0}_2$ s.t. $\vtheta^T\vx\ge\theta_0$. For the behavioral case, we used $\gamma=0.5$, and the optimization problem $\argmin_\vx cost=\norm{\vx-\vx_0}_2$ s.t. $\vw(\vtheta)^T\vx\ge\theta_0$. 

In Figure~\ref{fig:BR-illustration-quad-cost}, for agents' response to the algorithm, we calculated the agents that can afford the response with a budget of $B=100$ and performed an optimization problem on only those agents. We solved a cost minimization problem for each agent in the band specified by Lemma~\ref{lemma:quad-cost-band}: $\argmin_\vx cost=\sum_i c_i(x_i-x_{0, i})^2$ s.t. $\vtheta^T\vx\ge\theta_0$. For the behavioral case, we used $\gamma=0.5$, and the optimization problem $\argmin_\vx cost=\sum_i c_i(x_i-x_{0, i})^2$ s.t. $\vw(\vtheta)^T\vx\ge\theta_0$. 

In Figure~\ref{fig:BR-illustration-lin-cost}, for agents' response to the algorithm, we calculated the agents that can afford the response with a budget of $B=100$ and performed an optimization problem on only those agents. We solved a cost minimization problem for each agent in the band specified by Lemma~\ref{lemma:manhattan-cost-band}: $\argmin_\vx cost=\vc^T|\vx-\vx_0|$ s.t. $\vtheta^T\vx\ge\theta_0$. For the behavioral case, we used $\gamma=0.5$, and the optimization problem $\argmin_\vx cost=\vc^T|\vx-\vx_0|$ s.t. $\vw(\vtheta)^T\vx\ge\theta_0$. 
\section{Piece-wise Cost Function Solution}\label{sec:app-piece-wise-sol}
Consider a setting similar to the piece-wise cost function described. To decide the feature to spend $B_1$ of the budget, we are comparing $\frac{c_1}{\theta_1}$, $\frac{c_1}{\theta_2}$, and $\frac{c_1}{\theta_3}$ as they all have the same cost for the first step of the budget. Without loss of generality imagine we have $\frac{c_1}{\theta_1} < \frac{c_1}{\theta_2} < \frac{c_1}{\theta_3}$, therefore, we choose to allocate the $B_1$ amount of our budget to the first feature. For $B_2$, we do a similar comparison but we have to use $c_2$ for the first feature since the first feature is now in the second step, i.e., we compare $\frac{c_2}{\theta_1}$, $\frac{c_1}{\theta_2}$, and $\frac{c_1}{\theta_3}$. This could lead to resulting in investing in another feature, for example, if we have $\frac{c_1}{\theta_2} < \frac{c_2}{\theta_1} < \frac{c_1}{\theta_3}$, we would choose the second feature and invest $B_2$ in that feature. We continue this reasoning until we have reached the boundary. We designed our user study so the participants did not need to calculate if they reached the decision boundary and had all participants spend all their budgets. 

As seen in Figure~\ref{fig:2d-approx-illustration}, the quadratic cost movement differs from norm-2 movement, which moves the point to the closest point on the decision boundary. The piece-wise function we use for our user study is similar to a quadratic cost function with $c_2=0.85c_1$ and a decision boundary $0.78x_1+0.22x_2=70$ for the two-dimensional case. 

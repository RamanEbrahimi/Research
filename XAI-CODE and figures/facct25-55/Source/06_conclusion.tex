\section{Conclusion}\label{sec:conclusion}
We presented a new strategic classification framework that accounts for the cognitive biases (in the form of probability weighing functions) of strategic agents when assessing feature importance. 
{Our hypothesis of humans' behavioral biases manifesting as weight misperceptions was motivated by commonly studied Prospect Theoretical models \cite{kahnemann1979prospect}, which found that individuals tend to overweigh or underweigh certain outcomes when making decisions under risk. While not directly a notion of risk, we hypothesized that these biases may also arise when feature importance weights are interpreted by users.}
{Given this type of behavioral bias,} we{, first theoretically, } identify conditions under which the agents over- or under-invest in different features, the impacts of this on a firm's choice of classifier, and the impacts on the firm's utility and agents' welfare. Then, through a user study, we provide support for our theoretical model and results, showing that most participants respond sub-optimally when provided with an explanation of feature importance/contribution, over-investing in the most important feature, while under-investing in the least important one, {a behavior explainable by our proposed weight misperceptions model}. {That said, we recognize that the study of behavioral biases in decision-making is a rich area, and exploring alternative forms of bias beyond weight or threshold misperception (in both theory and user studies) is an exciting avenue for future research; we hope our work provides a first step in this direction.}
Exploring analytical models accounting for biases beyond misperception of feature weights {such as misperception in the cost function (or alternative firm cost functions, or misalignment between users/firm's perception of others' costs) which would not only alter the underlying movement trajectory of each agent (Fig.~\ref{fig:2d-approx-illustration}) but would also affect the constraint $c(\vx,\vx_0)\le B$, complicates the analysis. More general costs such as non-distance-based costs}, and exploring the possibility of designing explanation methods that can help mitigate biases, are also important directions for further investigation. 
